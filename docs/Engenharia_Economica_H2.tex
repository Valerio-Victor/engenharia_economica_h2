% Options for packages loaded elsewhere
\PassOptionsToPackage{unicode}{hyperref}
\PassOptionsToPackage{hyphens}{url}
\documentclass[
]{book}
\usepackage{xcolor}
\usepackage{amsmath,amssymb}
\setcounter{secnumdepth}{5}
\usepackage{iftex}
\ifPDFTeX
  \usepackage[T1]{fontenc}
  \usepackage[utf8]{inputenc}
  \usepackage{textcomp} % provide euro and other symbols
\else % if luatex or xetex
  \usepackage{unicode-math} % this also loads fontspec
  \defaultfontfeatures{Scale=MatchLowercase}
  \defaultfontfeatures[\rmfamily]{Ligatures=TeX,Scale=1}
\fi
\usepackage{lmodern}
\ifPDFTeX\else
  % xetex/luatex font selection
\fi
% Use upquote if available, for straight quotes in verbatim environments
\IfFileExists{upquote.sty}{\usepackage{upquote}}{}
\IfFileExists{microtype.sty}{% use microtype if available
  \usepackage[]{microtype}
  \UseMicrotypeSet[protrusion]{basicmath} % disable protrusion for tt fonts
}{}
\makeatletter
\@ifundefined{KOMAClassName}{% if non-KOMA class
  \IfFileExists{parskip.sty}{%
    \usepackage{parskip}
  }{% else
    \setlength{\parindent}{0pt}
    \setlength{\parskip}{6pt plus 2pt minus 1pt}}
}{% if KOMA class
  \KOMAoptions{parskip=half}}
\makeatother
\usepackage{longtable,booktabs,array}
\usepackage{calc} % for calculating minipage widths
% Correct order of tables after \paragraph or \subparagraph
\usepackage{etoolbox}
\makeatletter
\patchcmd\longtable{\par}{\if@noskipsec\mbox{}\fi\par}{}{}
\makeatother
% Allow footnotes in longtable head/foot
\IfFileExists{footnotehyper.sty}{\usepackage{footnotehyper}}{\usepackage{footnote}}
\makesavenoteenv{longtable}
\usepackage{graphicx}
\makeatletter
\newsavebox\pandoc@box
\newcommand*\pandocbounded[1]{% scales image to fit in text height/width
  \sbox\pandoc@box{#1}%
  \Gscale@div\@tempa{\textheight}{\dimexpr\ht\pandoc@box+\dp\pandoc@box\relax}%
  \Gscale@div\@tempb{\linewidth}{\wd\pandoc@box}%
  \ifdim\@tempb\p@<\@tempa\p@\let\@tempa\@tempb\fi% select the smaller of both
  \ifdim\@tempa\p@<\p@\scalebox{\@tempa}{\usebox\pandoc@box}%
  \else\usebox{\pandoc@box}%
  \fi%
}
% Set default figure placement to htbp
\def\fps@figure{htbp}
\makeatother
\setlength{\emergencystretch}{3em} % prevent overfull lines
\providecommand{\tightlist}{%
  \setlength{\itemsep}{0pt}\setlength{\parskip}{0pt}}
\usepackage[]{natbib}
\bibliographystyle{plainnat}
% PREÂMBULO PARA O PPC EM BOOKDOWN

% Codificação e fonte
\usepackage[T1]{fontenc}
\usepackage[utf8]{inputenc}
\usepackage{lmodern}

% Página e margens (A4, margens semelhantes a livro técnico)
\usepackage[a4paper,margin=2.5cm]{geometry}

% Espaçamento entre linhas e parágrafos
\usepackage{setspace}
\onehalfspacing

\setlength{\parindent}{1.25cm}
\setlength{\parskip}{6pt}

% Títulos de capítulo sem a palavra "Chapter"
\usepackage{titlesec}

\titleformat{\chapter}[display]
  {\normalfont\huge\bfseries} % estilo do título
  {\thechapter}               % só o número do capítulo (ex.: 1, 2, 3)
  {1ex}
  {}

\titlespacing*{\chapter}{0pt}{-10pt}{20pt}


% Título do sumário em português
\renewcommand{\contentsname}{SUMÁRIO}

% Título do bibliografia em português
\renewcommand{\bibname}{REFERÊNCIAS}

% Para tabela
\usepackage{booktabs}
\usepackage{longtable}

\usepackage{pdflscape}
\usepackage{bookmark}
\IfFileExists{xurl.sty}{\usepackage{xurl}}{} % add URL line breaks if available
\urlstyle{same}
\hypersetup{
  pdftitle={INTRODUÇÃO À ANÁLISE DE VIABILIDADE FINANCEIRA DE PROJETOS DE HIDROGÊNIO},
  pdfauthor={Autor: Prof.~Victor Valerio},
  hidelinks,
  pdfcreator={LaTeX via pandoc}}

\title{INTRODUÇÃO À ANÁLISE DE VIABILIDADE FINANCEIRA DE PROJETOS DE HIDROGÊNIO}
\author{\textbf{Autor}: \emph{\href{http://lattes.cnpq.br/1989464933096922}{\textbf{Prof.~Victor Valerio}}}}
\date{\textbf{Última Atualização}: 03 de fevereiro de 2026}

\begin{document}
\maketitle

{
\setcounter{tocdepth}{1}
\tableofcontents
}
\chapter{INTRODUÇÃO}\label{introduuxe7uxe3o}

\chapter{PRINCÍPIOS DE FINANÇAS}\label{princuxedpios-de-finanuxe7as}

A análise de viabilidade financeira constitui um componente indissociável do processo de concepção, avaliação e implementação de projetos de engenharia. Ainda que um empreendimento apresente plena viabilidade técnica, contando com disponibilidade de tecnologias adequadas, domínio dos processos produtivos e conformidade com critérios normativos e operacionais, sua materialização concreta depende, em maior medida, da comprovação de sua viabilidade financeira.

Nesse sentido, pode-se afirmar que a viabilidade técnica seja condição necessária, mas não condição suficiente para a implementação de um projeto de engenharia, uma vez que a decisão de implementação do empreendimento está condicionada à capacidade desse de remunerar adequadamente o recurso monetário empregado, compatibilizando retorno esperado e respectivo nível de risco.

É também relevante ressaltar que a análise de viabilidade financeira não deve ser compreendida apenas como um instrumento de verificação ex post da atratividade de um projeto previamente definido.

Ao contrário, a análise de viabilidade financeira constitui ferramenta analítica que deve exercer influência direta sobre o processo de tomada de decisão técnica, muitas vezes sendo responsável pela reformulação do próprio projeto de engenharia. Resultados financeiros insatisfatórios podem motivar ajustes na escala de produção, na escolha tecnológica, na configuração operacional ou na estratégia de suprimento de insumos, com o objetivo de aprimorar simultaneamente o desempenho técnico e econômico do empreendimento.

Em virtude dessa interação necessária entre decisões de natureza técnica e financeira, consolidou-se na literatura o conceito de engenharia econômica, entendido como o campo do conhecimento dedicado à aplicação sistemática de técnicas de análise financeira à avaliação de projetos de engenharia.

Adicionalmente, assim como os projetos de engenharia são desenvolvidos com elevado grau de rigor técnico, apoiados em princípios científicos e procedimentos estruturados, as análises financeiras devem ser conduzidas com equivalente rigor metodológico. A avaliação da viabilidade financeira de um projeto não se restringe à estimação simplificada de custos e receitas, mas envolve definição criteriosa de premissas econômicas amplamente reconhecidos na literatura especializada.

Nesse contexto, o presente capítulo tem por finalidade introduzir de forma sistemática os fundamentos conceituais da análise de viabilidade financeira de projetos de engenharia, apresentando os principais instrumentos utilizados no processo de avaliação financeira de investimentos de projetos de engenharia.

Por se tratar de uma apostila de caráter aplicado, os conceitos apresentados ao longo deste capítulo na medida do possível serão associados a projetos de produção de hidrogênio de baixo carbono, com ênfase em sistemas baseados na eletrólise da água. Espera-se que, ao término deste capítulo, o leitor esteja capacitado a compreender e aplicar, de forma estruturada e tecnicamente consistente, os conceitos fundamentais da análise de viabilidade financeira no contexto de projetos de produção de hidrogênio de baixo carbono.

\section{Os X Princípios Básicos}\label{os-x-princuxedpios-buxe1sicos}

\textbf{Princípio 01 - O Dinheiro Muda de Valor ao Longo do Tempo}

A base conceitual de toda a matemática financeira --- e, por extensão, da análise de viabilidade financeira de projetos --- fundamenta-se em um princípio elementar que orienta o conjunto de técnicas desenvolvidas neste capítulo: o dinheiro muda de valor ao longo do tempo. Como consequência lógica imediata desse princípio, tem-se que quantias monetárias expressas em diferentes períodos de tempo não são diretamente comparáveis. Para que tal comparação seja possível, torna-se necessário realizar um procedimento de correção do valor do dinheiro, de modo a expressar todas as quantias monetárias em uma mesma referência temporal \citep{Assaf2017}.

Há diversas razões que explicam por que uma mesma unidade monetária não preserva seu valor ao longo do tempo. Uma das explicações mais recorrentes na literatura econômica é o fenômeno da inflação, compreendido como o aumento generalizado e persistente do nível de preços da economia. À medida que os preços de bens e serviços se elevam ao longo do tempo, o poder de compra associado a uma mesma unidade monetária se reduz. Essa perda de poder de compra constitui, portanto, uma das formas pelas quais se manifesta a alteração do valor do dinheiro ao longo do tempo.

Entretanto, a inflação não é a única razão para a variação do valor do dinheiro ao longo do tempo. Existe também uma explicação de natureza comportamental, relacionada às preferências intertemporais dos agentes econômicos. Em termos gerais, a possibilidade de consumir ou adquirir bens no presente é, ceteris paribus, preferível à postergação do consumo para o futuro. Dessa forma, postergar o uso de uma unidade monetária no presente implica um sacrifício, o qual deve ser compensado por um valor adicional no futuro. Esse mecanismo de compensação está na origem do conceito de remuneração do capital ao longo do tempo.

Seja em razão da variação do nível geral de preços --- associada à inflação ---, seja em decorrência da recompensa exigida pelos agentes econômicos para postergar o consumo --- associada às preferências intertemporais ---, torna-se necessário corrigir os valores monetários de um fluxo de caixa antes de proceder à sua análise econômica. Essa correção é realizada por meio da aplicação de uma taxa de juros, que permite expressar valores monetários observados em diferentes períodos em uma base temporal comum.

A partir dessas considerações iniciais, emergem alguns dos conceitos fundamentais que compõem o vocabulário básico da análise de viabilidade financeira, os quais serão recorrentes ao longo deste capítulo.

Fluxo de Caixa
Conjunto de entradas e saídas de recursos monetários estimadas para diferentes momentos do tempo, associadas às etapas de implantação, operação e desativação de um projeto.

Vida Útil do Projeto
Conceito associado à duração temporal do fluxo de caixa do empreendimento. Em projetos de engenharia, a vida útil do projeto é, em geral, definida a partir do período de operação do equipamento ou sistema considerado mais relevante para a atividade-fim do empreendimento.

Taxa de Juros
De forma geral, o juro pode ser entendido como a remuneração associada ao uso de uma quantia monetária em uma operação distinta do consumo imediato. Ao longo deste material, essa remuneração será tratada como remuneração do capital, sendo o capital definido como o valor monetário empregado em uma operação financeira ou em um projeto de investimento. Assim, a taxa de juros pode ser compreendida como um coeficiente que expressa a remuneração do capital ao longo de um determinado intervalo de tempo.

\chapter{MERCADO DE HIDROGÊNIO}\label{mercado-de-hidroguxeanio}

\chapter{ESTUDOS DE CASO}\label{estudos-de-caso}

\bibliography{packages.bib}

\end{document}
